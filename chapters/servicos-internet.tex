\chapter{Serviços e Internet}

Neste capítulo vamos implementar uma sincronização com servidor externo dos contatos da nossa
aplicação. Através de requisições HTTP seremos capazes de enviar e receber dados do \textit{smartphone}
para um servidor, bem como utilizar um recurso do Android chamado \texttt{Service}. Ele terá a habilidade
de executar tarefas em \textit{background} enquanto o usuário continua a utilizar o aparelho.

Faremos o seguinte, vamos inicialmente entender como funciona e criar um cliente HTTP simples
que irá enviar dados e receber respostas. Logo em seguida vamos ver como funciona um \texttt{Service}
e quais suas vantagens. Por fim juntaremos esses recursos para criar um servidor de sincronização
dos contatos.

\section{Fazendo requisições HTTP}

O Android vem com um cliente \gls{http} embutido provido pela Apache, o \texttt{HTTP Components}
(\url{http://hc.apache.org/}). Na página \url{http://hc.apache.org/poweredby.html} podemos
observar que o Android utiliza o \texttt{HttpCore} e o \texttt{HttpClient} 4.0 embutidos,
ou seja, não é preciso adicioná-los como bibliotecas para a utilização.

\subsection{Servidor de requisições HTTP}

Para iniciar criaremos uma página em \gls{php} para receber requisições HTTP e retornar dados.
A instalação do ambiente PHP no Ubuntu é bem simples. Abra um terminal e siga os passos a seguir:

\begin{flushleft}\texttt{\$ sudo su\\
\# apt-get install php5 apache2\\}
\end{flushleft}

Agora que temos um servidor HTTP e uma linguagem de \textit{script} para tratar requisições, vamos
a criação da página PHP. Ainda no terminal, faça:

\begin{flushleft}\texttt{\# cd /var/www\\
\# mkdir contatos\\
\# cd contatos\\
\# touch index.php\\
\# chown usuario.usuario index.php\\}
\end{flushleft}

\paragraph{Observação}

Note que na última linha troque \texttt{usuario.usuario} pelo seu nome de
usuário seguido de ponto e o nome do seu grupo (que por padrão é o mesmo nome de usuário). Isso vai
permitir que você possa editar o arquivo sem ter que ser super usuário (\textit{root}).

Edite o arquivo com o seguinte conteúdo.

% index.php
\begin{listing}[H]
  \inputminted[linenos=true,frame=bottomline,tabsize=3]{ php }{ source/index-1.php }
  \caption{script PHP para tratar requisições [index.php]}
\end{listing}

Esse trecho de código funciona da seguinte maneira. A página \texttt{index.php} recebe um parâmetro
chamado \texttt{saudacao} [linha \circled{11}]. Em seguida um teste é realizado para saber qual a
saudação e onde é determinado qual resposta será devolvida [linhas \circled{13} a \circled{22}].
Para a resposta é utilizada uma tecnologia chamada \gls{json}. Mais detalhes no Apêndice \ref{appx:json}.

% TODO: criar aplicativo saudacao que acessa a página index.php

\subsection{App que realiza acesso externo}

Com o servidor implementado precisamos criar um aplicativo Android capaz de acessar nossa página.
No eclipse, crie um novo projeto Android chamado \textbf{Saudacao} usando nome do pacote
\inlinecode{saudacao.app} e um \texttt{Activity} chamado \texttt{MainActivity}.

Inicialmente criaremos uma classe \texttt{Requisicao} contendo o método público \inlinecode{obterResposta},
além dos métodos privados \inlinecode{enviar}, \inlinecode{lerResposta} e \inlinecode{parseJSON}.

% Requisicao.java
\begin{listing}[H]
  \inputminted[linenos=true,frame=bottomline,tabsize=3]{ java }{ source/Requisicao-1.java }
  \caption{Classe para realizar requisições HTTP [Requisicao.java]}
\end{listing}

Uma coisa chama bastante atenção neste trecho de código. Nossa \inlinecode{URL} usa o \texttt{IP
10.0.2.2} que no Android refere-se ao \texttt{localhost}. Isso porque cada instância do emulador
roda sob um roteador/\textit{firewall} virtual que o isola das interfaces e configurações de rede
da máquina e da \textit{internet}.

Mais detalhes sobre esse assunto podem ser encontrados em
\url{http://developer.android.com/tools/devices/emulator.html#emulatornetworking}.

Vamos a implementação do método \inlinecode{enviar}.

% Requisicao.java
\begin{listing}[H]
  \inputminted[linenos=true,frame=bottomline,tabsize=3]{ java }{ source/Requisicao-2.java }
  \caption{Implementação do método enviar [Requisicao.java]}
\end{listing}

Em seguida temos o método \inlinecode{lerResposta}.

% Requisicao.java
\begin{listing}[H]
  \inputminted[linenos=true,frame=bottomline,tabsize=3]{ java }{ source/Requisicao-3.java }
  \caption{Implementação do método lerResposta [Requisicao.java]}
\end{listing}

\newpage

Por fim o método \inlinecode{parseJSON}.

% Requisicao.java
\begin{listing}[H]
  \inputminted[linenos=true,frame=bottomline,tabsize=3]{ java }{ source/Requisicao-4.java }
  \caption{Implementação do método parseJSON [Requisicao.java]}
\end{listing}

Precisamos implementar o \textit{layout} que será usado pelo \texttt{MainActivity}.

% activity_main.xml
\begin{listing}[H]
  \inputminted[linenos=true,frame=bottomline,tabsize=3]{ xml }{ source/activity_main-1.xml }
  \caption{layout para envio e recebimento de mensagens [activity\b{ }main.xml]}
\end{listing}

Esse é um layout bem simples contendo apenas um \texttt{TextView} para mostrar a resposta,
um \texttt{EditText} para escrever a saudação a ser enviada e um \texttt{Button} que ao ser acionado
envia a saudação ao servidor e apresenta a resposta na tela.

Agora vem o \texttt{MainActivity}.

% MainActivity.java
\begin{listing}[H]
  \inputminted[linenos=true,frame=bottomline,tabsize=3]{ java }{ source/MainActivity-15.java }
  \caption{Utilização da classe \texttt{Requisicao} [MainActivity.java]}
\end{listing}

Execute o projeto e teste as saudações. Digite \texttt{oi} e \texttt{tchau} para obter respostas.
Com isso já somos capazes de enviar e obter dados de um servidor remoto.

\section{Tarefas em \textit{background} com \texttt{Service}}

Um \texttt{Service} é utilizado quando se quer executar tarefas que podem durar bastante tempo
e que não tenham interação com o usuário ou para responder a outras aplicações sobre algum serviço
que sua aplicação dispõe. Cada serviço deve ser regularmente mapeado no \texttt{AndroidManifest}
da mesma forma que os \texttt{Activity}'s.

Vamos voltar a implementar em cima da nossa aplicação \textbf{contatos} e criar um \texttt{Service}
que irá sincronizar os dados do celular com o servidor.

% http://www.felipesilveira.com.br/2010/05/content-providers/