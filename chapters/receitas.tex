\chapter{Livro de Receitas}

\section{Mostrando Diálogos}

No Android, podemos criar diálogos no \texttt{Activity} mostrando opções
ao usuário, como por exemplo, escolher itens de uma lista, ou responder
sim ou não a uma ação, etc.

Vamos incrementar algumas partes de nosso código e tentar encaixar
algumas funcionalidades relacionadas.

\subsection{Editar/Excluir ao clicar e segurar na ListView}

Vamos implementar uma ação comum no mundo Android, que é a seguinte: ao
clicar e segurar num item da \texttt{ListView}, ele mostra opções editar
e excluir, por exemplo. Isto pode ser feito facilmente usando
\texttt{AlertDialog.Builder}, uma classe com métodos pré-prontos para
serem usados por você.

Neste exemplo, precisaremos editar \texttt{ContatoHelper} e adicionar um
método para deletar um contato, editar nosso \texttt{MainActivity} no
método configurar e adicionar um \emph{Listener} que ao clicar e segurar
num item da \texttt{ListView} um método é acionado. Vamos a
implementação:

\begin{listing}[H]
  \inputminted[linenos=true,frame=bottomline,tabsize=3]{ java }{ source/ContatoHelper-5.java }
  \caption{Deletar dados existentes [ContatoHelper.java]}
\end{listing}

\begin{listing}[H]
  \inputminted[linenos=true,frame=bottomline,tabsize=3]{ java }{ source/MainActivity-8.java }
  \caption{Adicionar Listener para click longo [MainActivity.java]}
\end{listing}

Note a necessidade de um novo método em \texttt{MainActivity}, o
exibirMensagem. Ele é bastante útil quando se quer exibir uma mensagem
rapidamente e depois ela suma. Para isso usamos a classe \texttt{Toast}.

\subsubsection{Interface como parâmetro de um método}

Você já deve ter notado o uso de \emph{interface}'s como parâmetro dos
métodos, por exemplo na linha \circled{4} e \circled{11} do código
acima. Essa prática obriga ao programador implementar a classe na
passagem dos parâmetros.

Essa ideia vem de algumas linguagens de programação que possuem funções
como parâmetros para outras funções. Como o Java não suporta essa
característica, a solução veio em forma de uma \emph{interface}, a qual
o programador é obrigado a implementar seus métodos. Com isso o método
que recebe a \emph{interface} como parâmetro sabe exatamente o que ela
tem disponível.

A partir dessa observação, podemos justificar o uso da palavra reservada
final em alguns parâmetros dos métodos acima. Isso acontece porque
alguns parâmetros são utilizados dentro da implementação das
\emph{interface}'s.

Caso haja a necessidade de utilizar uma implementação em outra classe
você pode criar uma classe que implementa uma \emph{interface}, por
exemplo a \emph{interface} \texttt{OnItemLongClickListener}. Daí para a
passagem do parâmetro apenas crie uma instância da classe. Por exemplo,
suponha que você tenha uma classe chamada \texttt{OpcoesContato} que
implementa \texttt{OnItemLongClickListener}, nesse caso a linha
\circled{4} se tornaria:

\texttt{listView.setOnItemLongClickListener(new OpcoesContato());}

\subsection{Diálogo de confirmação}

Deletar dados é uma ação que deve ser feita com cuidado, então sempre é
bom confirmar com o usuário se ele deseja realmente deletar, no nosso
caso, um contato. Para isso usaremos o \texttt{AlertDialog.Builder} mais
uma vez, agora apenas com uma mensagem e os botões \emph{Sim} ou
\emph{Não}.

Ainda em \texttt{MainActivity} criaremos um outro
\texttt{AlertDialog.Builder} no momento que o usuário clicar em
\texttt{Deletar}. Segue o trecho:

\begin{listing}[H]
  \inputminted[linenos=true,frame=bottomline,tabsize=3]{ java }{ source/MainActivity-9.java }
  \caption{Diálogo de confirmação ao deletar contato [MainActivity.java]}
\end{listing}

Pronto, agora o trecho que deleta o contato foi movido para dentro do
\emph{Listener} do botão \emph{Sim}. No botão \emph{Não} passamos
\texttt{null} no \emph{Listener}, pois caso seja a opção escolhida
apenas fazemos nada. Você pode, se quiser, criar um \emph{Listener} e
mostrar uma mensagem do tipo, \emph{Cancelado pelo usuário}, para isso
usando o método \texttt{exibirMensagem}.

\subsection{Entrada de diferentes tipos de dados}

O Android foi desenvolvido com muitos recursos pré-prontos para
facilitar o desenvolvimento de aplicações. Um recurso bastante útil é a
distinção dos dados que irão ser inseridos nos \texttt{TextView}'s. Com
isso o teclado virtual do cliente se adapta ao tipo de dado que será
inserido. No nosso caso faremos distinção do campo \texttt{telefone},
onde apenas números e hífens (-) podem ser inseridos, e o campo
\texttt{e-mail} onde a presença do arroba (@) e pontos (.) são elementos
essenciais.

Vejamos alguns valores aceitos pelo \texttt{inputType}:

\begin{itemize}
\item
  Para textos:
  \begin{itemize}
  \item
    \texttt{text}
  \item
    \texttt{textCapCharacters}
  \item
    \texttt{textMultiLine}
  \item
    \texttt{textUri}
  \item
    \texttt{textEmailAddress}
  \item
    \texttt{textPersonName}
  \item
    \texttt{textPassword}
  \item
    \texttt{textVisiblePassword}
  \end{itemize}
\item
  Para números:
  \begin{itemize}
  \item
    \texttt{number}
  \item
    \texttt{numberSigned}
  \item
    \texttt{numberDecimal}
  \item
    \texttt{phone}
  \item
    \texttt{datetime}
  \item
    \texttt{date}
  \item
    \texttt{time}
  \end{itemize}
\end{itemize}
Precisaremos alterar apenas o \texttt{salvar.xml} localizado em
\texttt{res/layout}. Localize o atributo \texttt{inputType} dos campos
\texttt{telefone} e \texttt{e-mail} e altere os valores da seguinte
maneira:

\begin{listing}[H]
  \inputminted[linenos=true,frame=bottomline,tabsize=3]{ xml }{ source/salvar-2.xml }
  \caption{Distinção de dados [res/layout/salvar.xml]}
\end{listing}

\subsection{Validação de dados}

Mesmo configurando um \texttt{inputType} para seu \texttt{TextView} pode
não ser o bastante para que os dados inseridos estejam corretos. Para
isso usaremos a classe \texttt{Patterns} do pacote
\texttt{android.util}. Nela podemos encontrar alguns objetos bastante
úteis na hora de validar dados. Entre eles estão os objetos
\texttt{Patterns.EMAIL ADDRESS} e \texttt{Patterns.PHONE}. Com eles
podemos validar de forma simples os dados inseridos em nosso formulário.

Em nosso \texttt{SalvarActivity} adicionaremos um método validar
passando como parâmetro um \texttt{ContentValues}. Copie o método
\texttt{exibirMensagem} da classe \texttt{MainActivity} para mostrar uma
mensagem caso alguma validação seja falsa.

\paragraph{OBS:}

Para um melhor reuso crie uma classe abstrata que implementa o método
\texttt{exibirMensagem} e que extenda de \texttt{Activity} e faça com
que seus \texttt{Activity}'s herdem dela. É uma boa prática.

Vamos ao trecho de código:

\begin{listing}[H]
  \inputminted[linenos=true,frame=bottomline,tabsize=3]{ java }{ source/SalvarActivity-4.java }
  \caption{Validação dos dados [SalvarActivity.java]}
\end{listing}

\subsection{Fazendo uma ligação}

Já que estamos fazendo uma lista de contatos nada melhor que usar o
número do telefone dos contatos inseridos para realizar chamadas. Para
isso vamos aprender um pouco sobre \textbf{Permissões}.

Permissões no Android são definidas no \texttt{AndroidManifest.xml}. Ao
instalar seu aplicativo, o usuário saberá quais as permissões que o seu
aplicativo necessita para ser executado.

Por padrão, o Android traz uma série de permissões que auxiliam seu
aplicativo a se comunicar com o aparelho. Abaixo alguns exemplos:

\begin{itemize}
\item
  Verificação
  \begin{itemize}
  \item
    \texttt{ACCESS\_NETWORK\_STATE}
  \item
    \texttt{ACCESS\_WIFI\_STATE}
  \item
    \texttt{BATTERY\_STATS}
  \end{itemize}
\item
  Comunicação
  \begin{itemize}
  \item
    \texttt{BLUETOOTH}
  \item
    \texttt{CALL\_PHONE}
  \item
    \texttt{INTERNET}
  \item
    \texttt{SEND\_SMS}
  \end{itemize}
\end{itemize}
A lista completa pode ser vista em
\url{http://developer.android.com/reference/android/Manifest.permission.html}.

Edite o \texttt{AndroidManifest.xml} e adicione a permissao
\texttt{CALL\_PHONE}.

\begin{listing}[H]
  \inputminted[linenos=true,frame=bottomline,tabsize=3]{ xml }{ source/AndroidManifest-3.xml }
  \caption{Permissão de realizar chamadas [AndroidManifest.xml]}
\end{listing}

Agora vamos adicionar um item ao diálogo que aparece ao clicar e segurar
um item da \texttt{ListView}. Ele servirá para implementarmos o trecho
que realiza a chamada. Vamos a ele:

\begin{listing}[H]
  \inputminted[linenos=true,frame=bottomline,tabsize=3]{ java }{ source/MainActivity-10.java }
  \caption{Item chamar no diálogo [MainActivity.java]}
\end{listing}

Nesse trecho de código podemos ver o uso de \texttt{Intent}'s do prórpio
Android, nesse caso o Intent.ACTION\_CALL (veja linha \circled{14}). Ele
serve para chamar uma \texttt{Activity} que realize ligações. Atente
apenas para um detalhe - esse \texttt{Intent} faz a chamada sem
confirmação. Caso você queira que o usuário possa visualizar o número no
discador use o \texttt{Intent} Intent.ACTION\_DIAL. Faça esse teste e
veja a diferença entre os \texttt{Intent}'s.

Veja mais detalhes em
\url{http://developer.android.com/reference/android/content/Intent.html}.

\subsection{Enviando e-mail}

Para envio de e-mail você pode simplesmente usar a aplicação de e-mail
padrão do aparelho. Seguindo o mesmo princípio do exemplo anterior vamos
apenas inserir um trecho de código no método configurar da classe
\texttt{MainActivity}:

\begin{listing}[H]
  \inputminted[linenos=true,frame=bottomline,tabsize=3]{ java }{ source/MainActivity-11.java }
  \caption{Item enviar e-mail no diálogo [MainActivity.java]}
\end{listing}

Ao testar no emulador você receberá a mensagem: \textbf{No applications
can perform this action}. Traduzindo quer dizer que: Nenhuma aplicação
pode executar esta ação. Em outras palavras, nenhum cliente de e-mail
foi encontrado.

\section{Forçando região para teste}

Para podermos testar as \texttt{strings} de \emph{i18n} podemos forçar o
\texttt{Activity} a utilizar uma determinada linguagem. Isso se dá por
meio da classe \texttt{Locale}. Façamos um teste com o
\texttt{SalvarActivity} inserindo o trecho de código abaixo no método
\texttt{onCreate}. Vamos a ele:

\begin{listing}[H]
  \inputminted[linenos=true,frame=bottomline,tabsize=3]{ java }{ source/SalvarActivity-5.java }
  \caption{Forçando região [SalvarActivity.java]}
\end{listing}

Para visualizar a mudança crie \emph{strings} no seu arquivo
\texttt{strings.xml}. Substitua as \emph{strings} \texttt{Nome},
\texttt{Telefone}, \texttt{E-mail} e \texttt{Salvar} pelos respectivos
valores em inglês \texttt{Name}, \texttt{Phone}, \texttt{E-mail} e
\texttt{Save}. Agora crie outro arquivo \texttt{strings.xml} dentro do
diretório \texttt{/res/values-pt-rBR} e insira as mesmas \emph{strings}
citadas anteriormente, traduzindo cada valor.

Faça testes comentando a chamada para a função \texttt{forceLocale} e
veja as mudanças.

\subsection{Forçando região pelo emulador}

A maneira mais rápida e prática de forçar a região é pelo próprio
emulador. Vá até a lista de aplicativos e procure por
\texttt{Custom Locale}. Depois pesquise por \texttt{pt\_BR} e caso não
encontre clique em \texttt{Add New Locale}. Digite \texttt{pt\_BR} e
clique em \texttt{Add and Select}.

\section{Utilizando as Preferências do Android}

O Android já disponibiliza uma maneira de criar preferências de forma
fácil. Para demonstrar implementaremos um exemplo bem amplo, que irá nos
ajudar a entender ainda mais de Android. Para começar adicionaremos um
nova coluna a nossa tabela \texttt{contato} chamada \texttt{grupo}.
Depois adicionaremos um \emph{array} de \emph{string}'s ao nosso arquivo
\texttt{strings.xml} e ainda vamos aprender a utilizar um
\texttt{Spinner}, também conhecido como \emph{combo box}. Por último, e
não menos importante, usaremos as preferências para tornar padrão um
valor de nosso \texttt{Spinner}.

\subsection{Atualizando colunas de uma tabela \label{sssec:update-ddl}}

Como visto em \ref{ssec:model}, a classe \texttt{SQLiteOpenHelper}
obriga-nos a implementar os métodos \texttt{onCreate} e
\texttt{onUpgrade}. Neste ponto será necessário o uso do método
\texttt{onUpgrade}. Ele serve, como o nome sugere, para atualizar a
\gls{ddl} do banco de dados. Isso é útil quando seu cliente já possui
uma versão do seu aplicativo instalada e ele quer apenas atualizar para
uma nova versão. Também será necessário adicionar a coluna
\texttt{grupo} nas \emph{queries}. Abra a classe \texttt{ContatoHelper}
em \texttt{contatos.app.model} e faça as modificações:

\begin{listing}[H]
  \inputminted[linenos=true,frame=bottomline,tabsize=3]{ java }{ source/ContatoHelper-6.java }
  \caption{Nova coluna grupo na base de dados [ContatoHelper.java]}
\end{listing}

Vemos neste exemplo o uso da classe \texttt{Log} do pacote
\texttt{android.util}. Ela possui apenas métodos estáticos, assim não
precisamos instanciar, apenas faça a chamada dos métodos. Temos:

\begin{itemize}
\item
  \texttt{Log.w()}: para mostrar \emph{warning}'s, ou seja, avisos.
\item
  \texttt{Log.e()}: para mensagens de erro.
\item
  \texttt{Log.d()}: para mensagens \emph{debug}.
\item
  \texttt{Log.i()}: para mensagens informativas.
\item
  \texttt{Log.v()}: para outras mensagens.
\end{itemize}
\begin{listing}[H]
  \inputminted[linenos=true,frame=bottomline,tabsize=3]{ java }{ source/ContatoHelper-7.java }
  \caption{Modificação nas queries [ContatoHelper.java]}
\end{listing}

\subsection{Array de Strings}

No arquivo de \emph{string}'s do Android é possível criar vários
recursos. Dentre eles temos Cor, Dimensão, Estilo/Tema. Usando a
ferramenta ADT, crie um \texttt{String Array} em \texttt{strings.xml}
dentro de \texttt{res/values} e adicione alguns itens para representar
os valores da coluna \texttt{grupo}, e outro \texttt{String Array} para
representar os índices:

\paragraph{Dica:}

você pode tentar implementar o trecho usando uma tabela do banco de
dados. A ideia é a mesma, neste caso não seria necessário o uso de
\texttt{String Array}'s.

\begin{listing}[H]
  \inputminted[linenos=true,frame=bottomline,tabsize=3]{ xml }{ source/strings-1.xml }
  \caption{Array de Strings [strings.xml]}
\end{listing}

\subsection{Spinner, diálogo de seleção}

O \emph{Spinner} é ideal quando temos que escolher entre valores fixos,
sejam eles estáticos ou dinâmicos. Nosso exemplo irá utilizar valores
estáticos para popular o mesmo. Para isso utilizaremos o
\texttt{array\_grupos} que criamos em \texttt{res/values/strings.xml}.
Também veremos um exemplo de uso da classe \texttt{android.R} como visto
em \ref{par:r} em que é explicado a diferença entre as classes de
recursos. Mas antes temos que atualizar nosso \emph{layout}
\texttt{salvar.xml}. Adicione o \emph{Spinner} logo abaixo do
\texttt{e-mail}, como mostra o trecho abaixo:

\begin{listing}[H]
  \inputminted[linenos=true,frame=bottomline,tabsize=3]{ xml }{ source/salvar-3.xml }
  \caption{Adicionando elemento Spinner [res/layout/salvar.xml]}
\end{listing}

Agora já podemos carregar e popular o \emph{Spinner} na classe
\texttt{SalvarActivity}.

\begin{listing}[H]
  \inputminted[linenos=true,frame=bottomline,tabsize=3]{ java }{ source/SalvarActivity-6.java }
  \caption{Utilização de Spinner [SalvarActivity.java]}
\end{listing}

Note a utilização da classe \texttt{android.R} nas linhas \circled{10} e
\circled{11}. Eles servem para definir o \emph{layout} do
\emph{Spinner}. Isso quer dizer que você pode implementar como seu
\emph{Spinner} irá aparecer na tela da mesma maneira que implementamos a
linha da \texttt{ListView} em \ref{ssec:listview}.

\subsection{A classe PreferenceActivity}

Afinal vamos utilizar as preferências do Android. Neste exemplo a
usaremos para decidir qual grupo do \texttt{array\_grupos} aparecerá
selecionado por padrão. A princípio é um exemplo bem simples, mas que
pode ser ajustado para outras finalidades, o que importa realmente é a
ideia.

Para começar criaremos um \emph{layout} em \texttt{res/layout} chamado
\texttt{preferencias.xml}. No projeto clique com botão direito do
\emph{mouse} e selecione \texttt{New} $\rightarrow$ \texttt{Other...},
pesquise por \texttt{Android XML File} e \texttt{Next}. Em
\texttt{Resource Type} escolha \texttt{Preference} e escreva
\texttt{preferencias} em \texttt{File}. Logo abaixo em
\texttt{Root Element} escolha a opção \texttt{PreferenceScreen}, então
\texttt{Finish}.

Utilizando a ferramenta ADT adicione um elemento \texttt{ListPreference}
a \texttt{PreferenceScreen}. Defina os parâmetros necessários como
mostra o código abaixo:

\begin{listing}[H]
  \inputminted[linenos=true,frame=bottomline,tabsize=3]{ xml }{ source/preferencias-1.xml }
  \caption{XML descrevendo layout de preferências [res/xml/preferencias.xml]}
\end{listing}

Crie uma nova classe chamada \texttt{EditarPreferencias} em
\texttt{contatos.app.view} herdando de \texttt{PreferenceActivity}.
Agora de uma maneira bem simples implementaremos essa classe. Veja:

\begin{listing}[H]
  \inputminted[linenos=true,frame=bottomline,tabsize=3]{ java }{ source/EditarPreferencias-1.java }
  \caption{Activity para mostrar preferências [EditarPreferencias.java]}
\end{listing}

Para chamar a nova \texttt{Activity} temos ainda que mapeá-la no
\texttt{AndroidManifest} e criar um item no menu.

\begin{listing}[H]
  \inputminted[linenos=true,frame=bottomline,tabsize=3]{ xml }{ source/AndroidManifest-4.xml }
  \caption{Mapeando Activity EditarPreferencias [AndroidManifest.xml]}
\end{listing}

\begin{listing}[H]
  \inputminted[linenos=true,frame=bottomline,tabsize=3]{ xml }{ source/main_menu-2.xml }
  \caption{Adicionar item Preferências ao menu principal [res/menu/main_menu.xml]}
\end{listing}

Agora que adicionamos um item ao menu, temos que capturar o evento
quando o usuário o selecionar e direcioná-lo às Preferências. Isso deve
ser feito em \texttt{MainActivity}.

\begin{listing}[H]
  \inputminted[linenos=true,frame=bottomline,tabsize=3]{ java }{ source/MainActivity-12.java }
  \caption{Ir para Preferências pelo menu principal [MainActivity.java]}
\end{listing}

Note que para ter um código mais eficiente e otimizado tivemos que mudar
o método \texttt{irParaSalvar} para \texttt{irPara} passando como
parâmetro a classe que desejamos ir. Essa mudança é boa mais causa um
impacto em outros trechos do código. Conserte-os da seguinte maneira:

\begin{listing}[H]
  \inputminted[linenos=true,frame=bottomline,tabsize=3]{ java }{ source/MainActivity-13.java }
  \caption{Mudança em método irParaSalvar [MainActivity.java]}
\end{listing}

Por fim temos que selecionar o item que o usuário quer que esteja
selecionado por padrão ao inserir um novo contato. Assim, em
\texttt{SalvarActivity} adicione o trecho:

\begin{listing}[H]
  \inputminted[linenos=true,frame=bottomline,tabsize=3]{ java }{ source/SalvarActivity-7.java }
  \caption{Obtem o valor padrão definido nas Preferências [SalvarActivity.java]}
\end{listing}

\section{Grupo de Contatos usando Grid}

Uma das coisas mais legais quando falamos de aparelhos móveis é a ideia
da visão da lista de aplicativos usada comumente com o ícone e o texto
logo abaixo. Essa ideia pode ser facilmente implementada em um
aplicativo Android usando \texttt{GridView}.

Nessa implementação vamos criar uma tela que mostra os grupos de
contatos em forma de \emph{Grid} e ao clicar levaremos o usuário a lista
de contatos mostrando apenas aqueles contatos de um determinado grupo.

\subsection{Layout usando GridView}

Para começar criaremos um \emph{layout} em \texttt{res/layout} chamado
\texttt{grupos\_item.xml}. Ele irá conter a imagem e o texto que serão
exibidos no \texttt{GridView}. Faça como mostra o trecho abaixo:

\begin{listing}[H]
  \inputminted[linenos=true,frame=bottomline,tabsize=3]{ xml }{ source/grupos_item-1.xml }
  \caption{Item do Layout de Grupos [res/layout/grupos\b{ }item.xml]}
\end{listing}

Hora de criar o \texttt{GridView}. Para isso crie um novo \emph{layout}
em \texttt{res/layout} chamado \texttt{grupos.xml}. Adicione apenas um
\texttt{GridView} como mostra o trecho de código abaixo:

\begin{listing}[H]
  \inputminted[linenos=true,frame=bottomline,tabsize=3]{ xml }{ source/grupos-1.xml }
  \caption{Layout de Grupos [res/layout/grupos.xml]}
\end{listing}

\paragraph{Dica:}

a ferramenta ADT provê uma forma de pré-visualizar seu \emph{layout}.
Note que na linha 12 temos um comentário e nele temos a referência ao
\emph{layout} \texttt{grupos\_item}. Para isso apenas clique com botão
direito do \emph{mouse} na \texttt{GridView} e na opção
\texttt{Preview Grid Content} $\rightarrow$ \texttt{Choose Layout...}
selecione \texttt{grupos\_item}.

\subsection{Activity para visualizar os Grupos}

Como é de se imaginar temos que criar uma \texttt{Activity} para
visualizar os Grupos.

\begin{listing}[H]
  \inputminted[linenos=true,frame=bottomline,tabsize=3]{ java }{ source/GruposActivity-1.java }
  \caption{Activity para visualizar Grupos [GruposActivity.java]}
\end{listing}

Temos que criar duas classes internas para nos ajudar a criar cada item
do grupo de contatos. Para isso usaremos a classe abstrata
\texttt{BaseAdapter}.

\begin{listing}[H]
  \inputminted[linenos=true,frame=bottomline,tabsize=3]{ java }{ source/GruposActivity-2.java }
  \caption{Adapter responsável por cada item do Grid [GruposActivity.java]}
\end{listing}

Nesse momento precisamos usar a ferramenta Inkscape e criar alguns
ícones. Para os exemplos a seguir você deve criar um ícone para cada
item do grupo, sendo eles:

\begin{itemize}
\item
  amigos
\item
  trabalho
\item
  conhecidos
\item
  família
\end{itemize}
Para título de exemplo crie apenas ícones simples e depois tente fazer
itens mais sofisticados. Em
\url{http://developer.android.com/design/style/iconography.html} você
pode ver como devem ser criados os ícones para seu aplicativo.

\subsubsection{Criando ícones com Inkscape}

Use o Inkscape para criar um novo ícone. No menu \texttt{Arquivo}
$\rightarrow$ \texttt{Propriedades do Desenho...} ou apenas
\texttt{Shift + Ctrl + D} e altere a largura e altura para
\texttt{512px}.

Aperte \texttt{5} para centralizar a folha e crie um quadrado
(\texttt{F4}) um pouco menor que a página. Utilize \texttt{Ctrl} para
criar um quadrado perfeito. Altere a borda usando o círculo branco no
canto superior direito do quadrado. Selecione uma cor legal.

O Android possui uma paleta de cores que pode lhe ajudar inicialmente.
Veja a tabela abaixo:

\ctable[caption = Paleta de cores do Android,
pos = H, center, botcap]{llc}
{% notes
}
{% rows
\FL
\parbox[b]{0.13\columnwidth}{\raggedright
\textbf{Cor}
} & \parbox[b]{0.42\columnwidth}{\raggedright
\textbf{Tom claro}
} & \parbox[b]{0.46\columnwidth}{\centering
\textbf{Tom escuro}
}
\ML
\parbox[t]{0.13\columnwidth}{\raggedright
Azul
} & \parbox[t]{0.42\columnwidth}{\raggedright
\texttt{\#33B5E5} \squarecolor{android-blue}
} & \parbox[t]{0.46\columnwidth}{\centering
\texttt{\#0099CC} \squarecolor{android-dark-blue}
}
\\\noalign{\medskip}
\parbox[t]{0.13\columnwidth}{\raggedright
Roxo
} & \parbox[t]{0.42\columnwidth}{\raggedright
\texttt{\#AA66CC} \squarecolor{android-purple}
} & \parbox[t]{0.46\columnwidth}{\centering
\texttt{\#9933CC} \squarecolor{android-dark-purple}
}
\\\noalign{\medskip}
\parbox[t]{0.13\columnwidth}{\raggedright
Verde
} & \parbox[t]{0.42\columnwidth}{\raggedright
\texttt{\#99CC00} \squarecolor{android-green}
} & \parbox[t]{0.46\columnwidth}{\centering
\texttt{\#669900} \squarecolor{android-dark-green}
}
\\\noalign{\medskip}
\parbox[t]{0.13\columnwidth}{\raggedright
Laranja
} & \parbox[t]{0.42\columnwidth}{\raggedright
\texttt{\#FFBB33} \squarecolor{android-orange}
} & \parbox[t]{0.46\columnwidth}{\centering
\texttt{\#FF8800} \squarecolor{android-dark-orange}
}
\\\noalign{\medskip}
\parbox[t]{0.13\columnwidth}{\raggedright
Vermelho
} & \parbox[t]{0.42\columnwidth}{\raggedright
\texttt{\#FF4444} \squarecolor{android-red}
} & \parbox[t]{0.46\columnwidth}{\centering
\texttt{\#CC0000} \squarecolor{android-dark-red}
}
\LL
}

Mais detalhes em
\url{http://developer.android.com/design/style/color.html}.

Para alterar a cor clique com botão direito do \emph{mouse} no quadrado
e selecione \texttt{Preenchimento e contorno}. Observe a entrada de
texto onde aparece \texttt{RGBA}. Altere com os valores acima mantendo
os dois últimos, pois eles são referentes a transparência
(\emph{alpha}).

Chegou a hora de exportar seu ícone para os tamanhos sugeridos pelo
Android. Basta ir no menu \texttt{Arquivo} $\rightarrow$
\texttt{Exportar Bitmap...} ou ainda \texttt{Shift + Ctrl + E}. Os
tamanhos estão definidos na tabela abaixo:

\ctable[caption = Localização e tamanho dos ícones,
pos = H, center, botcap]{ll}
{% notes
}
{% rows
\FL
\parbox[b]{0.31\columnwidth}{\raggedright
\textbf{Local}
} & \parbox[b]{0.17\columnwidth}{\raggedright
\textbf{Tamanho}
}
\ML
\parbox[t]{0.31\columnwidth}{\raggedright
\texttt{res/drawable-xhdpi}
} & \parbox[t]{0.17\columnwidth}{\raggedright
\texttt{96px}
}
\\\noalign{\medskip}
\parbox[t]{0.31\columnwidth}{\raggedright
\texttt{res/drawable-hdpi}
} & \parbox[t]{0.17\columnwidth}{\raggedright
\texttt{72px}
}
\\\noalign{\medskip}
\parbox[t]{0.31\columnwidth}{\raggedright
\texttt{res/drawable-mdpi}
} & \parbox[t]{0.17\columnwidth}{\raggedright
\texttt{48px}
}
\\\noalign{\medskip}
\parbox[t]{0.31\columnwidth}{\raggedright
\texttt{res/drawable-ldpi}
} & \parbox[t]{0.17\columnwidth}{\raggedright
\texttt{36px}
}
\LL
}

Exporte o ícone para cada um desses diretórios, crie-os caso não
existam. Como temos quatro grupos crie quatro ícones usando cores
diferentes. Siga a nomenclatura sugerida em \ref{sssec:nomeicones}
Convenção de nomes para ícones, exemplo:
\texttt{ic\_launcher\_grupo\_amigos.png}.

\subsection{Implementando o Adapter}

\begin{listing}[H]
  \inputminted[linenos=true,frame=bottomline,tabsize=3]{ java }{ source/GruposActivity-3.java }
  \caption{implementação do Adapter [GruposActivity.java]}
\end{listing}

Como visto em \ref{ssec:listview} Mostrando os dados na View, no
\texttt{Adapter} podemos fazer \emph{cache} dos objetos e otimizar o
código. Isso pode ser observado a partir da linha \circled{25} até a
linha \circled{35}, onde um teste é realizado para ver se a linha está
em \emph{cache}.

\paragraph{Observação:}

na linha \circled{37} existe um trecho de código que não está nada
otimizado. No entanto usando \texttt{string-array} é a única maneira de
dar certo. Isso poderia ser evitado se os grupos de contatos fossem
retirados do banco de dados. Seguindo as instruções antes abordadas
tente você mesmo implementar usando banco de dados. É uma ótima maneira
de aprender melhor como funciona um aplicativo Android.

Finalize adicionando o \texttt{Activity} no
\texttt{AndroidManifest.xml}. Clique na aba inferior em
\texttt{Application} e em \texttt{Application Nodes} clique em
\texttt{Add}. Escolha \texttt{Activity} na lista de opções e no atributo
\texttt{Name} clique em \texttt{Browser} e busque por
\texttt{GruposActivity}.

Para visualizar a nova \texttt{Activity} é preciso adicionar um novo
item no menu principal. Reveja \ref{ssec:act} Activity, e implemente
essa parte. Não esqueça de adicionar uma condição no método
\texttt{onOptionsItemSelected} da classe \texttt{MainActivity}.

\subsection{Selecionando contatos de um determinado grupo}

Para não deixar dúvidas quanto a implementação deste trecho vamos fazer
com que ao clicar em um determinado grupo, somente contatos daquele
grupo apareçam na lista que fica em \texttt{MainActivity}.

Primeiro sobrescreva o método listar da classe \texttt{ContatoHelper}
para que ele receba um parâmetro, que irá representar o grupo.

\begin{listing}[H]
  \inputminted[linenos=true,frame=bottomline,tabsize=3]{ java }{ source/ContatoHelper-8.java }
  \caption{Método listar com parâmetro grupo [ContatoHelper.java]}
\end{listing}

A implementação do clique em um item do \emph{grid} é semelhante a vista
em \ref{ssec:edit} Editando dados existentes, onde criamos uma variável
contendo o \emph{namespace} do nosso parâmetro.

Copie o método \texttt{irPara} da classe \texttt{MainActivity}, mudando
apenas o primeiro parâmetro de \texttt{intent.putExtra} para o novo
\emph{namespace}, na linha \circled{17}.

Por fim, inclua o método \texttt{configurar} em \texttt{onCreate}, o
qual será responsável por configurar para onde ir ao clicar em um item
do \emph{grid}.

\begin{listing}[H]
  \inputminted[linenos=true,frame=bottomline,tabsize=3]{ java }{ source/GruposActivity-4.java }
  \caption{Evento de clique em um item do \textit{grid} [GruposActivity.java]}
\end{listing}

Agora é preciso capturar o parâmetro em \texttt{MainActivity}. Para isso
basta fazer como descrito abaixo:

\begin{listing}[H]
  \inputminted[linenos=true,frame=bottomline,tabsize=3]{ java }{ source/MainActivity-14.java }
  \caption{Captura de parâmetro vindo de \texttt{GruposActivity} [MainActivity.java]}
\end{listing}

Muito bem, eis que temos um aplicativo já bem completo. Se tudo deu
certo você deve ter uma tela como vemos abaixo:

\begin{figure}[h]
    \centering
    \includegraphics[scale=0.6]{img/screenshot-grupos.png}
    \caption{Tela de Grupos}
\end{figure}
