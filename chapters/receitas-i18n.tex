\section{Internacionalização (i18n)}

\subsection{Forçando região para teste}

Para podermos testar as \texttt{strings} de i18n podemos forçar o \texttt{Activity}
a utilizar uma determinada linguagem. Isso se dá por meio da classe \texttt{Locale}.
Façamos um teste com o \texttt{SalvarActivity} inserindo o trecho de código abaixo no
método \texttt{onCreate}. Vamos a ele:

% SaveActivity.java
\begin{listing}[H]
  \inputminted[linenos=true,frame=bottomline,tabsize=3]{ java }{ source/SalvarActivity-5.java }
  \caption{Forçando região [SalvarActivity.java]}
\end{listing}
% Força locale para uma determinada regiao, bom para testar i18n
% Locale locale = new Locale("pt", "BR");
% Locale.setDefault(locale);
% Configuration configuration = new Configuration();
% configuration.locale = locale;
% getBaseContext().getResources().updateConfiguration(configuration, getBaseContext().getResources().getDisplayMetrics());

Para visualizar a mudança crie \textit{strings} no seu arquivo \texttt{strings.xml}. Substitua
as \textit{strings} \texttt{Nome}, \texttt{Telefone}, \texttt{E-mail} e \texttt{Salvar} pelos
respectivos valores em inglês \texttt{Name}, \texttt{Phone}, \texttt{E-mail} e \texttt{Save}.
Agora crie outro arquivo \texttt{strings.xml} dentro do diretório \texttt{/res/values-pt-rBR} e
insira as mesmas \textit{strings} citadas anteriormente, traduzindo cada valor.

Faça testes comentando a chamada para a função \texttt{forceLocale} e veja as mudanças.

% TODO: mostrar o uso de array de strings.